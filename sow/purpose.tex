\قسمت{هدف}
نیروی کار آموزش دیده‌ی آماده ورود به بازار کار اقدام به جستجوی شغل متناسب با توانایی ها و علاقه مندی های خود می کند، اما به دلیل عدم آگاهی کافی نیروی کار از نیازمندی های بخش صنعت و عدم آگاهی کارفرمایان از وضعیت نیروی کار فرآیند کاریابی دچار مشکل می شود. مراکز کاریابی دانشگاهی به عنوان یکی از راه کارهای حل این مشکل می توانند به عنوان رابط بین متقاضیان نیروی کار و کارجویان عمل نمایند. مرکز کارآفرینی دانشگاه صنعتی شریف در راستای تسریع جذب فارغ التحصیلان به بازار کار اقدام به راه اندازی پرتال کاریابی شریف نموده است که مشاوره و هدایت فارغ التحصیلان به محل کار پایدار، شناسایی محل های جذب و ثبت تقاضای شغل از طرف فارغ التحصیلان و پیشنهاد دهی به کارجویان و برعکس از وظایف آن می باشد، اما به علت برآورده نشدن انتظارات، اهداف تعیین شده سامانه موجود و عدم استقبال فراگیر از سوی جامعه فارغ التحصیلان جویای کار و صاحبان صنعت، مرکز کارآفرینی دانشگاه شریف تصمیم به برگزاری مناقصه ای برای برون سپاری تحلیل، طراحی و پیاده سازی سامانه کاریابی جدیدی کرده است.  گروه تحلیل گران طراح سیستم با تحلیل و بررسی نقاط ضعف و قوت سامانه های کاریابی داخلی و خارجی تصمیم به شرکت در مناقصه موجود گرفته است تا با تحلیل، طراحی و پیاده سازی سامانه کاریابی جدید در کنار برخورداری از کیفیت مناسب و پیاده سازی قابلیت های جدید، انتظارات مرکز کارآفرینی شریف را محقق سازد و در جذب استقبال مخاطبان موفق عمل نماید.  در گام های بعدی این سامانه به صورت اختصاصی در حوزه فناوری اطلاعات و شغل های مرتبط با این حوزه گسترش خواهد یافت و در چشم انداز آن پیاده سازی قابلیت های مختص به این حوزه در سامانه، پیش بینی شده است.
