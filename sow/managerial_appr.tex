\قسمت{رویکرد مدیریتی}
\زیرقسمت{ملاحظات تشکیل تیم}
انجام پروژه‌ی \PRJN{}  به نحو احسن، نیازمند آشنایی و تسلّط  گروه بر مفاهیم و علوم گوناگونی در حوزه‌ی فناوری اطلاعات و مهندسی نرم‌افزار است. بنا بر این، در انتخاب اعضای گروه \GRPN{} مواردی را مدّ نظر قرار دادیم که برخی از آن‌ها عبارتند از:
\شروع{itemize}
\فقره مدیریت پروژه و فرآیند
\فقره مهندسی نیازها
\فقره تحلیل مسائل و راه‌حل‌ها و طراحی آن‌ها
\فقره پایگاه داده‌ها
\فقره برنامه‌نویسی وب
\فقره متدهای توسعه‌ی نرم‌افزار
\فقره متدهای سنجش نرم‌افزار
\پایان{itemize}
همچنین، عواملی مانند توانایی کار گروهی، نظم، پایبندی به مقرّرات و موارد مشابه نیز از نیازمندی‌های هر کار گروهیست.

با در نظر گرفتن موارد فوق، تلاش کردیم تا گروهی تشکیل دهیم که بتواند نیازهای فوق را پوشش بدهد. به این ترتیب، اعضای گروه \GRPN{} به این شرح انتخاب شدند:
\شروع{description}
\فقره[سید مهران خلدی] دانشجوی کارشناسی مهندسی نرم‌افزار دانشگاه صنعتی شریف
\فقره[هادی ستوده] دانشجوی کارشناسی فناوری اطلاعات دانشگاه صنعتی شریف
\فقره[آرمان رهبر] دانشجوی کارشناسی فناوری اطلاعات دانشگاه صنعتی شریف
\پایان{description} 

\زیرقسمت{مدیریت و تجربیات}
مدیریت تیم به عهده‌ی آقای سید مهران خلدی خواهد بود. مختصری از سوابق ایشان در ادامه آمده:
\شروع{itemize}
\فقره سوابق کاری در شرکت «\href{http://bayan.co.ir/}{گروه راهبرد بیان} \hfill اسفند ۹۱ - کنون
\شروع{itemize}
\فقره مدیریت پروژه‌ی سامانه‌ی برگزاری مسابقات برنامه‌نویسی بیان \hfill تیر ۹۲ - کنون
\فقره مدیر امنیتی محصولات بیان \hfill تیر ۹۲ - کنون
\فقره سرپرست کمیته‌ی علمی دومین مسابقه‌ی بین‌المللی برنامه‌نویسی بیان \hfill بهمن ۹۲
\فقره برنامه‌نویس جاوا در پروژه‌های «\href{http://salam.ir/}{موتور متاجستجوی سلام}» و «\href{http://zal.ir/}{موتور جستجوی زال}» 
\فقره برنامه‌نویسی \چبر{frontend} در پروژه‌ی «\href{http://hod.ir/}{میل‌سرور هد}»
\پایان{itemize}

\فقره سوابق تحصیلی
\شروع{itemize}
\فقره محصّل دوره‌ی کارشناسی مهندسی کامپیوتر (گرایش نرم‌افزار) در دانشگاه صنعتی شریف
\فقره فارغ‌التّحصیل رشته‌ی ریاضی-فیزیک از دبیرستان علّامه‌حلّی تهران
\پایان{itemize}

\فقره افتخارات
\شروع{itemize}
\فقره کسب رتبه‌ی دوم در دومین و سومین مسابقات هک و نفوذ شریف
\فقره کسب مدال نقره در ۲۲مین و ۲۳مین المپیاد جهانی کامپیوتر
\پایان{itemize}
\پایان{itemize}
رزومه‌ی کامل ایشان از طریق \href{http://www.semekh.ir/}{وبگاه ایشان} در دسترس است.


\زیرقسمت{آموزش‌های مورد نیاز}
ممکن است اعضاء تیم در برخی حوزه‌ها نیاز به کسب دانش بیشتر جهت انجام کارهای محوّله داشته باشند. با توجه به جلساتی که تشکیل شد، نیاز اوّلیه به یادگیری مطالب زیر احساس شد:
\شروع{itemize}
\فقره برنامه‌نویسی به زبان پایتون
\فقره تکنولوژی‌های وب (مانند \چبر{HTML} و \چبر{CSS})
\فقره برنامه‌نویسی تحت وب توسط چارچوب جنگو
\فقره کنترل نسخه توسّط \چبر{git}
\فقره کار با نرم‌افزار مدیریت پروژه \چبر{redmine}
\پایان{itemize}

\زیرقسمت{زمان‌بندی جلسات}
جلسات مورد نیاز پروژه‌ی \PRJN{} در دو دسته‌ی کلّی جای می‌گیرند:
\شروع{itemize}
\فقره جلسات درون گروهی: این جلسات هر هفته و مطابق با آیین‌نامه‌ی گروه برگزار می‌شود و در آن اعضای گروه علاوه‌بر ارائه‌ی گزارش پیشرفت کارها، به تبادل نظر و اصلاح جهت پیشرفت پروژه می‌پردازند.
\فقره جلسات برون‌گروهی: این جلسات، بسته به درخواست ناظر پروژه و به صورت ماهیانه برگزار می‌شود. در این جلسات، مدیریت گروه گزارشی از پیشرفت پروژه ارائه می‌کند، و با توجه به بازخورد ناظر، مسیر انجام پروژه را اصلاح می‌کند. نتیجه‌ی این جلسات در جلسات درون‌گروهی منعکس خواهد شد.
\پایان{itemize}

\زیرقسمت{فواصل و شیوه‌ی ارائه‌ی گزارش‌ها}
علاوه بر گزارش‌هایی که در جلسات ارائه می‌شود و در بخش قبل به آن‌ها اشاره شد، اعضای گروه موظفند تا در فاز پیاده‌سازی تمامی فعالیت‌های خود را از طریق نرم‌افزار \چبر{issue tracker} مستندسازی کنند. این مستندات به صورت برخط در دسترس ناظر پروژه خواهد بود و در صورت نیاز، ایشان می‌تواند توسّط \چبر{issue tracker} با توسعه‌دهندگان ارتباط برقرار کنند.

همچنین، در فاز تحلیل و طراحی، دو مرحله ارائه‌ی \چبر{Use Case} و \چبر{ERD} و \چبر{DFD} نیز موجود است، که در این مراحل نیز ناظر می‌تواند نظر خود را به اطلاعات طراحان برساند تا بررسی و اعمال شوند.

\زیرقسمت{مدیریت ناسازگاری‌ها}
در صورت بروز ناسازگاری میان اعضای گروه و یا میان ناظر و گروه، مراحل زیر دنبال خواهند شد:
\شروع{enumerate}
\فقره تلاش اوّلیه برای متقاعد کردن تمامی اعضای گروه و ناظران، و جلب رضایت آن‌ها می‌باشد.
\فقره در صورت بی‌ثمر ماندن گام پیشین، رأی‌گیری میان حضّار انجام خواهد شد.
\پایان{enumerate}

\زیرقسمت{مدیریت گستره}
در صورت درخواست \چبر{feature}های جدید (چه از سوی اعضای گروه، و چه از سوی کارفرما)، جهت جلوگیری از رخداد \چبر{feature creep}، اعضای گروه جلسه‌ای خواهند داشت که در آن به بررسی مزایا و معایب اضافه شدن یک قابلیت به پروژه می‌پردازند. در حالت کلّی، زمانی با افزایش گستره موافقت خواهد شد که پیاده‌سازی آن ارزش افزوده‌ای را به‌همراه داشته باشد و شرایط پروژه (مانند بودجه و زمان‌بندی) را نقض نکند.

نهایتاً؛ موافقت مدیر پروژه جهت تغییر گستره‌ی نرم‌افزار الزامیست.


