\قسمت{پیشینه}
\زیرقسمت{محرک‌های انجام پروژه}
با توجه به این که حجم زیاد مسافرت‌ها و تعداد زیاد هتل ها و تنوع قیمت‌های آن‌ها مسافران را در انتخاب هتل و رزرو آن‌ها با مشکلات زیادی روبرو کرده، مسأله‌ی انتخاب هتل مناسب و رزرو آن وجود دارد. این پروژه برای کمک به حل این مشکل تعریف شده تا مسافران بتوانند با جستجو و رزرو برخط، به سهولت هتل مناسب برای خود را انتخاب کنند. علاوه بر آن، صاحبان هتل‌ها از مزیت گزارش‌گیری‌های مناسب بهره‌مند خواهند شد و نسبت به دیگر هتل‌داران مزیت رقابتی رزرو آنلاین را دارند.


\زیرقسمت{تاریخچه}
با پیشرفت فناوری اطلاعات و انجام سریع کارها در خدمات برپایه‌ی وب، بسیاری از امکاناتی که در گذشته با مراجعه حضوری و بهره‌مندی از شناخت بسیار کم خدمت توسط مشتری انجام می‌شدند، هم‌اکنون با سرعت و بازدهی اقتصادی بیشتر قابل انجام هستند.
\\
هرچه شرکت‌ها زودتر به سمت استفاده از این فناوری بروند، منفعت‌های اقتصادی و محبوبیت تجاری بیشتری کسب می‌کنند.

\زیرقسمت{اهداف و نتایج مورد انتظار}
در این پروژه قصد داریم با طراحی یک سامانه‌ی تحت وب، به هتل‌داران و مسافران کمک کرده و فرآیند رزرواسیون را تسهیل کنیم. سیمرغ بستری را برای بهبود خدمات هتل‌داری و تعامل هتل‌داران و مسافران فراهم می‌آورد.
\\
انتظار داریم با پیاده‌سازی این بستر:
\شروع{itemize}
\فقره میزان فروش هتل‌هایی که در سیمرغ ثبت می‌شوند افزایش یابد.
\فقره زمان و میزان زحمات مسافران در فرآیند رزرواسیون به‌طور قابل توجهی کاهش یابد.
\فقره هتل‌داران از وضعیت اقتصادی خود گزارش‌های تحلیلی داشته باشند.
\پایان{itemize}

\زیرقسمت{توصیف محصول نهایی}
\شروع{itemize}
\فقره وجود قسمت های ثبت نام جداگانه برای کارجویان و کارفرمایان
\فقره امکان احراز هویت کارفرمایان 
\فقره وجود فرمهای ثبت مهارت ها توسط کارجویان
\فقره امکان انتخاب زمینه کاری متناسب با کارجویان
\فقره امکان افزودن مهارت های جدید توسط کارجویان 
\فقره وجود فرم های ثبت شرایط فرد مورد نیاز توسط کارفرمایان
\فقره امکان بارگذاری آگهی جذب کارمند 
\فقره امکان بارگذاری رزومه توسط کارجویان
\فقره امکان مشاهده رزومه ها توسط کارفرمایان به تفکیک مهارت های تخصصی و میزان تحصیلات
\فقره ارائه جدید ترین اطلاعیه های جذب کارمند در صفحه اول به تفکیک زمینه کاری
\فقره امکان جستجوی آگهی جذب کارمند بر اساس حرفه و محل کار 
\فقره امکان رتبه‌بندی مناسب موقعیت‌های شغلی بنا به توانایی‌های فرد
\فقره ارائه بخش پرسش و پاسخ
\فقره پیشنهاد دادن کارجویان واجد شرایط به هر کار فرما از طریق رایانامه  
\فقره پیشنهاد دادن کارهای متناسب با شرایط هر کارجو  از طریق رایانامه
\فقره امکان ارسال دعوتنامه به مصاحبه حضوری از کارفرما به کارجویان مورد نظر
\پایان{itemize}

