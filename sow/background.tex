\قسمت{زمینه}
\زیرقسمت{علت انجام پروژه}
امروزه مساله اشتغال در جوامع مختلف و به ویژه کشور ما یک مساله حساس و مهم برای جوانان و فارغ التحصیلان رشته های مختلف محسوب می شود. در این میان فارغ التحصیلان رشته های مرتبط با فناوری اطلاعات(IT) می توانند در زمینه های مختلف و متنوعی مشغول به کار شوند. اما فرایند کاریابی فرایندی زمان بر و پر هزینه است. با توجه به این مشکلات این فرصت برای ما به عنوان دانشجویان این رشته وجود دارد که  در صورت وجود سرمایه گذار مناسب،با طراحی و پیاده سازی یک درگاه کاریابی در حوزه فناوری اطلاعات، فرایند کاریابی برا ی فارغ التحصیلان این رشته را تسریع و تسهیل کنیم و از هزینه های آن بکاهیم. بدین ترتیب افراد جویای کار در حوزه فناوری اطلاعات در نقاط مختلف کشور می توانند با مراجعه به این درگاه با کارفرمایان ارتباط برقرار کرده و در شغل متناسب با تخصصشان مشغول به کار شوند. در این میان در دانشگاه صنعتی شریف یک درگاه کاریابی به آدرس jobs.sharif.ir  وجود دارد که از واسط کاربری مناسبی برخوردار نیست و خدمات خود را به نحو مناسبی ارائه نمی دهد. با توجه قسمت های مشترک در این سامانه و سامانه کاریابی فناوری اطلاعات می توانیم ضمن طراحی سامانه اصلی ، درگاه کاریابی دانشگاه صنعتی شریف را نیز بهبود بخشیم.

\زیرقسمت{تاریخچه}
درگاه کاریابی یک وب سایت است که طراحی شده است تا کارفرمایان را قادر سازد  نیاز های کاری خود را اعلام کنند و افراد واجد شرایط بتوانند با کارفرمایان ارتباط برقرار کرده و در صورت توافق مشغول به کار شوند. 

اولین بار \چبر{Online Career Center} به عنوان یک موسسه غیر انتفاعی با پشتیبانی چهارده شرکت بزرگ ایجاد شد تا کارجویان را قادر سازد رزومه های خود را ارسال کنند. 

در سال 1994 شرکتی به نام \چبر{NetStart} نرم افزاری را به کمپانی های مختلف می فروخت که به آن ها کمک می کرد تا آگهی های کار خود را بر روی وب سایتشان قرار دهند و در خواست های رسیده از کارجویان را مدیریت کنند.

در سال 1995 شش روزنامه مهم با هم متحد شدند تا قسمت نیازمندی های خود را بر روی وب قرار دهند. آدرس این سامانه \چبر{Careerpath.com} بود و شامل روزنامه های \چبر{Los Angeles Times, The Boston Globe, Chicago Tribune, the New York Times, San Joes Mercury News, The Washington Post } بود.

امروزه  \چبر{Monster.com} یکی از بزرگرین درگاه های کاریابی در جهان محسوب می شود. در سال 2006 این درگاه یکی از 20 وب سایت پر بیننده در جهان شناخته شد. این درگاه در سال 1999 با همکاری دو شرکت \چبر{The Monster Board} و \چبر{Online Career Center} ایجاد شد. این در گاه به کارجویان کمک می کند تا کار متناسب حرفه و محل زندگی شان را پیدا کنند.در هر زمان این درگاه  بیش از یک میلیون آگهی کار دریافت می کند و بیش از یک میلیون رزومه در آن قرار داده می شود وبیش از 63 میلیون کارجو در ماه به آن مراجعه می کنند.در این کمپانی بیش از 5000 کارمند در 36 کشور جهان مشغول به کار هستند و ساختمان مرکزی آن در نیویورک قرار دارد.

\چبر{Workopolis.com}   یک درگاه کاریابی کانادایی است.  این درگاه ماهانه مورد بازدید 3 میلیون نفر قرار می گیرد. ساختمان مرکزی آن در تورنتو قرار دارد  وبا دو زبان فرانسوی و انگلیسی ارائه می شود. یک نسخه موبایل از این درگاه نیز در سال 2008 راه اندازی شد.  \چبر{Workopolis} در سال 2010 طراحی مجدد شد و یک راهبرد جدید را معرفی کرد.

چند درگاه موجود در داخل کشور هم عبارتند از: \چبر{jobcity.ir, karbank.ir, karyab.net }

اما در حوزه تخصصی کامپیوتر تعداد محدودی ازدرگاه های کاریابی وجود دارد. به عنوان مثال می توان به \چبر{computerjobs.com}  اشاره کرد که در سال 1995 در آمریکا راه اندازی شد و به سرعت در این کشور به یکی از محبوب ترین درگاه های کاریابی تبدیل شد. امروزه این درگاه دسترسی به مشاغل حوزه فناوری اطلاعات در نقاط مختلف جهان را ارائه می دهد.

\زیرقسمت{اهداف و نتایج مورد انتظار}
\شروع{itemize}
\فقره کاهش زمان صرف شده برای یافتن کار مورد نظر توسط کارجویان
\فقره کاهش هزینه های کاریابی و استخدام
\فقره دسترسی کارجویان به فرصت های شغلی مختلف در نقاط مختلف کشور
\فقره امکان انتخاب بهترین فرد برای کارفرمایان 
\فقره کاهش بیکاری در میان افراد جامعه و به کار گرفتن افراد خبره و با مهارت از میان آنها
\پایان{itemize}

\زیرقسمت{توصیف محصول نهایی}
\شروع{itemize}
\فقره وجود قسمت های ثبت نام جداگانه برای کارجویان و کارفرمایان
\فقره امکان احراز هویت کارفرمایان 
\فقره وجود فرمهای ثبت مهارت ها توسط کارجویان
\فقره امکان انتخاب زمینه کاری متناسب با کارجویان
\فقره امکان افزودن مهارت های جدید توسط کارجویان 
\فقره وجود فرم های ثبت شرایط فرد مورد نیاز توسط کارفرمایان
\فقره امکان بارگذاری آگهی جذب کارمند 
\فقره امکان بارگذاری رزومه توسط کارجویان
\فقره امکان مشاهده رزومه ها توسط کارفرمایان به تفکیک مهارت های تخصصی و میزان تحصیلات
\فقره ارائه جدید ترین اطلاعیه های جذب کارمند در صفحه اول به تفکیک زمینه کاری
\فقره امکان جستجوی آگهی جذب کارمند بر اساس حرفه و محل کار 
\فقره امکان رتبه‌بندی مناسب موقعیت‌های شغلی بنا به توانایی‌های فرد
\فقره ارائه بخش پرسش و پاسخ
\فقره پیشنهاد دادن کارجویان واجد شرایط به هر کار فرما از طریق رایانامه  
\فقره پیشنهاد دادن کارهای متناسب با شرایط هر کارجو  از طریق رایانامه
\فقره امکان ارسال دعوتنامه به مصاحبه حضوری از کارفرما به کارجویان مورد نظر
\پایان{itemize}

