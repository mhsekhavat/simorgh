\قسمت{پیشینه}
\زیرقسمت{محرک‌های انجام پروژه}
با توجه به این که حجم زیاد مسافرت‌ها و تعداد زیاد هتل ها و تنوع قیمت‌های آن‌ها مسافران را در انتخاب هتل و رزرو آن‌ها با مشکلات زیادی روبرو کرده، مسأله‌ی انتخاب هتل مناسب و رزرو آن وجود دارد. این پروژه برای کمک به حل این مشکل تعریف شده تا مسافران بتوانند با جستجو و رزرو برخط، به سهولت هتل مناسب برای خود را انتخاب کنند. علاوه بر آن، صاحبان هتل‌ها از مزیت گزارش‌گیری‌های مناسب بهره‌مند خواهند شد و نسبت به دیگر هتل‌داران مزیت رقابتی رزرو آنلاین را دارند.


\زیرقسمت{تاریخچه}
با پیشرفت فناوری اطلاعات و انجام سریع کارها در خدمات برپایه‌ی وب، بسیاری از امکاناتی که در گذشته با مراجعه حضوری و بهره‌مندی از شناخت بسیار کم خدمت توسط مشتری انجام می‌شدند، هم‌اکنون با سرعت و بازدهی اقتصادی بیشتر قابل انجام هستند.
\\
هرچه شرکت‌ها زودتر به سمت استفاده از این فناوری بروند، منفعت‌های اقتصادی و محبوبیت تجاری بیشتری کسب می‌کنند.

\زیرقسمت{اهداف و نتایج مورد انتظار}
در این پروژه قصد داریم با طراحی یک سامانه‌ی تحت وب، به هتل‌داران و مسافران کمک کرده و فرآیند رزرواسیون را تسهیل کنیم. سیمرغ بستری را برای بهبود خدمات هتل‌داری و تعامل هتل‌داران و مسافران فراهم می‌آورد.
\\
انتظار داریم با پیاده‌سازی این بستر:
\شروع{itemize}
\فقره میزان فروش هتل‌هایی که در سیمرغ ثبت می‌شوند افزایش یابد.
\فقره زمان و میزان زحمات مسافران در فرآیند رزرواسیون به‌طور قابل توجهی کاهش یابد.
\فقره اطمینان کاربران از خرید خود با استفاده از سیمرغ افزایش یابد.
\فقره هتل‌داران از وضعیت اقتصادی خود گزارش‌های تحلیلی داشته باشند.
\پایان{itemize}

\زیرقسمت{توصیف محصول نهایی}
\شروع{itemize}
\فقره وجود قسمت های ثبت نام جداگانه برای هتل‌داران و مشتریان
\فقره اهراز هویت هتل‌داران و اعطای دسترسی متفاوت به ایشان
\فقره امکان معرفی و مدیریت اطلاعات هتل‌ها توسط هتل‌داران
\فقره امکان دسته‌بندی امکانات هتل‌ها در درجه‌بندی‌های مختلف
\فقره امکان مشاهده و جستجوی درجه‌های مختلف امکانات هتل‌ها بصورت مصوّر توسط مشتریان
\فقره امکان نظارت بر محتوا و مشخصات هتل‌ها و تایید برای نمایش در نتایج جستجو توسط مدیر سایت
\فقره امکان ثبت درخواست رزرو نهایی برای بازه‌ی زمانی مشخص اتاق‌های جستجو شده توسط مشتریان در سیمرغ
\فقره تغییر وضعیت اتاق رزرو شده به «در انتظار پرداخت» با اتصال به سرور هتل مربوطه
\فقره اعطای فرصت ۳۰ دقیقه‌ای برای پرداخت هزینه‌ی اتاق رزرو شده و جلوگیری از رزرو سایرین در این فرصت
\فقره اطلاع رسانی به هتل‌دار از رزرو و وضعیت پرداختی و مشخصات مشتریان
\فقره امکان برقراری ارتباط و تعامل با وب‌سرویس هتل‌ها و به‌روز رسانی اطلاعات اتاق‌ها
\فقره امکان گزارش‌گیری هتل‌داران از درخواست‌های رزرو، وضعیت درخواست اتاق‌ها، درآمد کسب شده و سایر موارد انجام شده در سیمرغ
\فقره امکان محاسبه‌ی کارمزد مربوط به هر هتل‌دار و پرداخت آن توسط هتل‌دار
\فقره صفحات اختصاصی برای هر هتل شامل اطلاعات کلی، موقعیت جغرافیایی به همراه نقشه، اطلاعاتی از امکانات هتل و ...
\فقره امکان ثبت نظر و ارزیابی در مورد کیفیت امکانات هتل توسط مشتریان و مشاهده‌ی میانگین نتایج ارزیابی کاربران
\پایان{itemize}

