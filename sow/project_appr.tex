\قسمت{رویکرد پروژه}
\زیرقسمت{مسیر پروژه}
در تحلیل، طراحی و پیاده‌سازی این سامانه از متدولوژی چابک\زیرنویس{Agile} استفاده خواهیم کرد، که شامل ۴ فاز می‌شود:
\شروع{enumerate}
\فقره روش آبشاری\زیرنویس{Waterfall}\\
این فاز با استفاده از روش آبشاری انجام می شود.
\شروع{enumerate}
\فقره آغازش سیستم
\فقره تعریف گستره پروژه
\فقره تحلیل مسئله
\فقره تحلیل نیازمندی‌ها و انتظارات 
\فقره طراحی منطقی (‌نمودارهای \چبر{Use Case-DFD/ERD})
\فقره مشخص کردن راه حل‌های جایگزین و انتخاب بهترین 
\فقره تحلیل تصمیم
\فقره طراحی فیزیکی
\فقره پیاده سازی و آزمودن
\فقره نصب و راه اندازی
\پایان{enumerate}
خروجی این قسمت از پروژه علاوه بر شامل شدن واسط کاربری، نمونه اولیه پرتال     کاریابی شریف نیز می باشد.
    
\فقره روش تکراری\زیرنویس{Iterative}\\
در این قسمت پروژه از روش تکراری استفاده می کند که شامل موارد زیر است:
\شروع{enumerate}
\فقره تحلیل و طراحی و پیاده سازی به صورت همزمان 
\فقره گرفتن بازخورد ناظر پروژه در جلسه  مشترک تعیین شده توسط ایشان و رفتن به مرحله قبل
\پایان{enumerate}


\فقره روش تکراری\\
    در این قسمت پروژه از روش تکراری استفاده می کند که شامل موارد زیر است:
 \شروع{enumerate}
\فقره تحلیل و طراحی و پیاده سازی به صورت همزمان 
\فقره گرفتن بازخورد ناظر پروژه در جلسه  مشترک تعیین شده توسط ایشان و رفتن به مرحله قبل
\پایان{enumerate}
در این قسمت قابلیت‌هایی مختص حوزه فناوری اطلاعات را بر روی سامانه پیاده سازی شده در مراحل قبلی پیاده سازی می کنیم 
   
   
\فقره تحویل نهایی \\
پیاده سازی نهایی و آزمون سامانه به منظور رفع معایب و در آخر نصب و راه اندازی محصول نهایی
\پایان{enumerate}

فعالیت‌های زیر نیز در کلیه فاز‌ها به صورت مشترک خواهد بود.
\شروع{itemize}
\فقره جمع آوری نیازمندی‌ها و انتظارات
\فقره مستند سازی و ارائه دادن 
\فقره تحلیل امکان سنجی
\پایان{itemize}


\زیرقسمت{تحویل‌دادنی‌ها}
\شروع{itemize}
\فقره تحویل گزارش به صورت شرح داده شده  در بخش مسیر مدیریتی قسمت  فواصل و شیوه ارائه گزارش‌ها آورده شده است.
\فقره هر دو هفته یکبار اعضای گروه کدهای پیاده سازی شده خود را از طریق ابزارهای کنترل نسخه یکپارچه سازی می‌کنند.
\فقره در تاریخ ۹ آبان ۱۳۹۲ پروپوزال این پروژه تحویل مرکز کارآفرینی شریف می‌شود.
\فقره در تاریخ ۲۱ آبان ۱۳۹۲ بعد از موفقیت در برعهده گرفتن تحلیل، طراحی و پیاده سازی این سامانه، نمودار \چبر{Use Case} تحویل مدیر پروژه می‌شود.
\فقره در تاریخ ۷ آذر ۱۳۹۲ نمودار‌های جریان داده و رابطه موجودیت تحویل مدیرپروژه می‌شود.
\فقره در تاریخ ۲۰ آذر۱۳۹۲  در جلسه ای مشترک با ناظر پروژه از فاز اول پروژه بهره برداری خواهد شد.
\فقره در تاریخ ۴ دی ۱۳۹۲  در جلسه ای مشترک با ناظر پروژه از فاز دوم پروژه بهره برداری خواهد شد.
\فقره در تاریخ ۲ بهمن ۱۳۹۲  در جلسه ای مشترک با ناظر پروژه از فاز سوم پروژه بهره برداری خواهد شد.
\فقره در تاریخ ۹ بهمن ۱۳۹۲  در جلسه ای مشترک با ناظر پروژه پروژه به صورت رسمی افتتاح می‌گردد.
\پایان{itemize}