\قسمت{شروط رضایت‌مندی}
\زیرقسمت{معیار موفقیت}
\شروع{itemize}
\فقره نسخه نهایی درگاه تا تاریخ ۹ بهمن ۹۲ تحویل داده شود.
\فقره کل هزینه مصرفی حداکثر 100,000,000 ریال باشد.
\فقره درگاه ازسرعت مناسبی برخوردار باشد  و هموراه در دسترس باشد. 
\فقره واسط کاربری در نسخه نهایی کارکردی آسان  و مورد قبول داشته باشد.
\فقره سرمایه طبق زمان بندی تعیین شده به دست اعضای تیم برسد.
\فقره از درگاه پشتیبانی مناسب به عمل آید و امکان ایجاد تغییرات در آن وجود داشته باشد. 
\فقره درگاه بین کارجو و کارفرما نقش واسطه را داشته باشد و از ارتباط مستقیم جلوگیری شود.
\فقره در صورت تشخیص بزرگنمایی در ثبت مهارت ها توسط کارجویان از امتیاز آنها کاسته شود. 
\فقره در صورت جعلی بودن هویت کارفرمایان حساب کاربری آن ها بسته شود.
\پایان{itemize}

\زیرقسمت{فرض‌ها}
\شروع{itemize}
\فقره اعضای تیم مسلط به برنامه نویسی وب و مفاهیم پایگاه داده هستند.
\فقره پروژه در چهار فاز پیاده سازی می شود و هر فاز حداکثر 2 هفته به طول می انجامد.
\فقره هزینه صرف شده برای پروژه بین 80,000,000 ریال تا 120,000,000 ریال می باشد.
\فقره اعضای تیم حداقل 10 ساعت در هفته برای کار بر روی پروژه وقت می گذارند.
\فقره هر هفته اعضای تیم فعالیت های انجام شده را به مدیر گروه گزارش می دهند.
\فقره مهارت های ثبت شده توسط کارجویان واقعی است و بزرگنمایی نمی کنند.
\پایان{itemize}

\زیرقسمت{مخاطرات}
\شروع{itemize}
\فقره حذف درس توسط یکی از اعضای تیم
\فقره به تعویق افتادن پیاده سازی یکی از فاز ها 
\فقره  به وجود  آمدن هزینه های پیش بینی نشده
\فقره به تعویق افتادن تزریق سرمایه به اعضای تیم
\فقره از بین رفتن اطلاعات ذخیره شده
\فقره بزرگ نمایی در ثبت مهارت های کاربران
\فقره ارتباط مستقیم کارجو و کارفرما و از بین رفتن نقش درگاه
\فقره عدم تشخیص هویت های جعلی کارفرمایان
\فقره از دسترس خارج شدن درگاه
\فقره نفوذ غیر قانونی یه درگاه
\پایان{itemize}