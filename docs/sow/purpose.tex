\قسمت{هدف سند}
در این سند ما بر آن شدیم که با توجه به افزایش لحظه به لحظه‌ی مسافرت‌های داخلی، افزایش تعداد هتل‌ها با امکانات و قیمت‌های متنوع و همچنین تلاش در صدد ثبات قیمت‌ها با حذف واسطه‌گری‌های بی حد و مرز یک بستر مناسب برای ارتباط امن بین مسافران و هتل‌داران ارائه دهیم. در سند پیش رو با توجه به گستردگی بازار موجود در حوزه توریست و هتل‌داری در سرتاسر کشور تلاش بر آن شد که یک سامانه رزرواسیون هتل تحت وب پیاده‌سازی گردد تا در سراسر کشور و حتی بیرن از مرزهای جغرافیایی قابلیت دسترسی داشته باشد. این سامانه که سامانه رزرواسیون هتل سیمرغ نام دارد قرار است تعاملی مستقیم میان هتل‌داران و مسافران برقرار کند. به این صورت که هتل‌داران با اضافه کردن مشخصات کیفی و کمی هتل خود در سامانه اقبال خود برای بیشتر شدن مشتریانشان را افزایش می‌دهند. از طرف دیگر مسافران نیز که در سطوح مختلف می‌توانند به عنوان کاربران سامانه باشند می‌توانند هتل‌ها را متناسب با موقعیت و کیفیت و قیمت مشاهده و طبقه‌بندی نمایند و از نظرات سودمند دیگران بهره بجویند و گزینه مورد نظرشان را با توجه به نیازمندی‌ها و بودجه‌شان انتخاب کنند و در انتها با توجه به سطح دسترسی به صورت مستقیم در خود سامانه به رزرو هتل بپردازند و مبلغ را پرداخت کنند.
\newline
در حال حاضر با توجه به افزایش سطح دسترسی هرچه بیشتر اقشار جامعه به بسترهای شبکه و اینترنت به صورت ثابت و یا همراه، امید است که این سامانه مورد اقبال افراد جامعه قرار گیرد و حتی برای مسافرت‌های کوتاه مدت خود با صرفه جویی در وقت و هزینه به سمت راهکارهای برخط سوق داده شوند و به دنبال رزرواسیون به این گونه برآیند. سامانه سیمرغ با ایجاد فرآیند و تسهیلاتی که در ادامه‌ی این سند برای هتل‌داران و مسافران ذکر شده‌اند در صدد آن است که دریچه‌ای شود تا بخش عمده‌ای از درآمد صنعت گردشگری و توریست یعنی فراهم آوردن مکانی برای اقامت برای گردشگران به روشی‌ آسان از کانال این سامانه عبور کند.