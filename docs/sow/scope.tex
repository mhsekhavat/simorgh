\قسمت{گستره}
\زیرقسمت{ذینفعان سامانه}
\شروع{enumerate}
\فقره مالکان سامانه‌ی سیمرغ: به ازای هر تراکنش در سیستم یعنی رزرو شدن اتاق توسط مسافر و پرداخت مبلغ آن، درصدی از آن طبق ضوابط برای مالکان سامانه برداشته می‌شود و مابقی به هتل‌دار پرداخت می‌شود.
\فقره مسافران: هر مسافر می تواند با ثبت نام در سامانه پیش از سفر، با صرفه جویی در وقت و هزینه تمام گزینه‌ی مناسب جهت اقامتش را در شهر مقصد با جزئیات، به همراه تصاویر مشاهده کند و مناسب‌ترین را انتخاب و آن را رزرو نماید.
\فقره هتل‌داران: هتل‌دار بدون پرداخت هزینه‌های گزاف تبلیغاتی می‌تواند صرفا با ارائه خدمات مطلوب و جلب رضایت مشتریان پیشین با استفاده از این سامانه اقدام به جلب مشتریان از سراسر کشور نماید.
\فقره پذیرش هتل: به دلیل برخط شدن فرآیند خدمت‌رسانی، بسیاری از تماس‌های تلفنی گنگ و ابهام برانگیز توسط مسافران در جهت اطلاع از خدمات، قیمت و تعداد اتاق‌های خالی هتل که اغلب بیشتر وقت پذیرش را می‌گیرند حذف می‌گردند و جستجو در سامانه سیمرغ جایگزین آن می‌شود که این امر باعث افزایش کیفیت پذیرش هتل در خدمت‌رسانی در دیگر بخش‌ها خواهد شد.
\فقره ایجاد کنندگان و نگه‌دارندگان سامانه: تحلیل‌گران، طراحان، سازندگان و نگه‌دارندگان سامانه که آن را به وجود خواهند آورد با توجه به حقوق و سابقه‌ی کاری که به دست می‌آورند از زینفعان خواهند بود.
\فقره رسانه‌های تبلیغاتی: با توجه به نیاز به تبلیغات در رسانه‌های تبلیغاتی جهت همه‌گیر شدن سامانه، این رسانه‌ها با توجه به دریافت وجه برای تبلیغات به عنوان ذینفع محسوب می‌شوند.
\فقره حامیان مالی: به دلیل وجود بستر تحت وب حامیان مالی می‌توانند با انجام تبلیغات در سامانه خدمات یا کالای خود را به صورت تخصصی برای کاربران که عموما مسافران هستند معرفی کنند.
\فقره آژانس‌های مسافرتی: سامانه سیمرغ محیط مناسبی است برای آژانس‌های مسافرتی تا بلیط‌های عادی یا چارتری خود را در بسته‌ای شامل بلیط، هتل و خدمات توریستی ارائه دهند که هتل مورد نظرشان را متناسب با زمان بلیط رفت و برگشت تحت عنوان کاربر ویژه به همراه تخفیف از سامانه انتخاب کنند.
\فقره بانک طرف قرارداد سامانه: با توجه به اینکه رزرواسیون به صورت برخط در درون خود سامانه انجام خواهد شد و همچنین حجم بالای تبادلات مالی در آن، شعبه بانک طرف قرارداد زینفع محسوب می‌شود که باید به صورت هوشمندانه انتخاب گردد تا بعد بتوان از تسهیلات آن شعبه مانند وام و غیره استفاده‌های ویژه نمود.

\پایان{enumerate}

\زیرقسمت{داده‌ها}

داده هایی که در سامانه جریان دارند از منابع گوناگون بدست می‌آیند:
\شروع{itemize}
\فقره داده‌هایی که هتل‌دار در اختیار سامانه قرار می‌دهد:
\شروع{itemize}
\فقره اطلاعات حساب کاربری هتل شامل نام هتل، تعداد ستاره، موقعیت و غیره
\فقره اطلاعات شرح امکانات و خدمات هتل و کیفیت، کمیت و قیمت اتاق‌ها به همراه تصاویر
\فقره اطلاعات وضعیت اتاق‌های خالی در بازه‌های زمانی گوناگون
\فقره اطلاعات تماس با هتل
\پایان{itemize}
\فقره داده‌هایی که مسافر در اختیار سامانه قرار می‌دهد:
\شروع{itemize}
\فقره اطلاعات حساب کاربری مسافر
\فقره اطلاعات اتاق مورد نیاز جهت رزرو شامل موقعیت، تاریخ، نوع هتل، تعداد تخت و غیره
\فقره اطلاعات تماس
\پایان{itemize}
\فقره داده‌هایی که سامانه در اختیار کاربران قرار می‌دهد:
\شروع{itemize}
\فقره جستجو و پیشنهاد دهی هوشمند به مسافران جهت انتخاب هر چه بهتر هتل متناسب با نیازمندی‌هایشان
\فقره رده‌بندی هتل‌ها بر اساس تعداد جذب مسافر، میزان جلب رضایت مشتریان و غیره
\فقره ایجاد پل ارتباطی امن ما بین هتل و مسافر
\پایان{itemize}
\پایان{itemize}


\زیرقسمت{فرآیندها}
\شروع{itemize}

\فقره ایجاد و اصلاح حساب کاربری هتل‌دار
\فقره صحت‌سنجی حساب کاربری هتل‌دار
\فقره ایجاد و اصلاح هتل و اتاق‌ و ثبت کامل مشخصات آن‌ها
\فقره دریافت اطلاعات وضعیت اتاق‌ها
\فقره ایجاد و اصلاح حساب کاربری مسافر
\فقره جستجوی هوشمند با توجه به محدودیت‌ها و نیازمندی‌ها
\فقره پیشنهاد دهی هتل‌های با شرایط ویژه
\فقره ارسال به بانک جهت انجام تراکنش
\فقره صدور معرفی نامه مسافر به هتل
\فقره دریافت نظرات و بازخورد مسافران
\فقره ربات جستجوگر
\فقره صدور رتبه بندی هتل‌ها
\فقره صدور گزارشات گوناگون
\پایان{itemize}

\زیرقسمت{مکان‌ها}
با توجه بر بستر تحت وب، کاربران از تمام نقاط جهان توانایی دسترسی به سامانه سیمرغ را دارا خواهند بود و در صورت توانایی انجام تراکنش‌های بانکی با داخل کشور خواهند توانست عملیات رزرو را نیز انجام دهند. البته بیشتر کاربران مسافران داخلی - مخصوصا کلان شهرها - خواهند بود که در جستجوی مکانی مناسب جهت اقامت در  شهر مقصدشان می‌باشند.