\قسمت{رویکرد مدیریتی}
\زیرقسمت{ملاحظات تشکیل تیم}
برای اجرای پروژه، به نقش‌های زیر نیاز داریم. بدیهی است یک نفر می‌تواند بیشتر از یک نقش را ایفا کند و یک نقش را چند نفر می‌توانند ایفا کنند:
\شروع{itemize}
\فقره مدیر پروژه
\فقره تحلیل‌گر نیازمندی‌ها
\فقره تحلیل‌گر مسئله
\فقره طراح پایگاه داده
\فقره متخصص شبکه
\فقره مدیر سیستم
\فقره برنامه‌نویس
\پایان{itemize}

افراد زیر به عنوان اعضای ثابت گروه با معیارهای کار گروهی، منظم و ضابطه‌مند انتخاب شدند. ممکن است برای انجام بخش‌هایی از پروژه با سایر افراد قرارداد کوتاه‌مدت بسته شود:
\شروع{description}
\فقره[محمد حسین سخاوت] دانشجوی کارشناسی مهندسی نرم‌افزار دانشگاه صنعتی شریف
\فقره[رضا آل یاسین] دانشجوی کارشناسی مهندسی نرم‌افزار دانشگاه صنعتی شریف
\فقره[سجاد جلالی] دانشجوی کارشناسی مهندسی نرم‌افزار دانشگاه صنعتی شریف
\پایان{description} 

\زیرقسمت{مدیریت و تجربیات}
مدیریت تیم به عهده‌ی آقای محمد حسین سخاوت خواهد بود. مختصری از سوابق ایشان در ادامه آمده:
\شروع{itemize}
\فقره برنامه‌نویسی در شرکت «\href{http://bayan.co.ir/}{گروه راهبرد بیان}» 
\فقره مشارکت در پروژه‌های متن باز آکادمیک از جمله \چبر{Mars}

\فقره سوابق تحصیلی
\شروع{itemize}
\فقره محصّل دوره‌ی کارشناسی مهندسی کامپیوتر (گرایش نرم‌افزار) در دانشگاه صنعتی شریف
\فقره فارغ‌التّحصیل رشته‌ی ریاضی-فیزیک از دبیرستان روزبه تهران
\پایان{itemize}

\فقره افتخارات
\شروع{itemize}
\فقره کسب رتبه‌ی اول در مسابقات کشوری آزمایشگاهی-کارگاهی رایانه سال ۱۳۸۹
\فقره کسب مدال طلا در المپیاد کشوری دانش‌آموزی کامپیوتر سال ۱۳۸۹
\فقره کسب مدال برنز در المپیاد دانش‌آموزی کامپیوتر حوزه‌ی آسیا سال ۱۳۸۹
\فقره کسب رتبه‌ی هشتم در نوزدهمین المپیاد دانشجویی کشوری رشته‌ی کامپیوتر سال ۱۳۹۳
\پایان{itemize}
\پایان{itemize}
رزومه‌ی کامل ایشان از طریق \href{https://www.linkedin.com/profile/view?id=159934426}{پروفایل \چبر{linkedin} ایشان} در دسترس است.


\زیرقسمت{آموزش‌های مورد نیاز}
ممکن است اعضاء تیم در برخی حوزه‌ها نیاز به کسب دانش بیشتر جهت انجام کارهای محوّله داشته باشند. با توجه به جلساتی که تشکیل شد، نیاز اوّلیه به یادگیری مطالب زیر احساس شد:
\شروع{itemize}
\فقره برنامه‌نویسی به زبان پایتون
\فقره تکنولوژی‌های وب (مانند \چبر{HTML} و \چبر{CSS})
\فقره برنامه‌نویسی تحت وب توسط چارچوب جنگو
\فقره کنترل نسخه توسّط \چبر{git}
\فقره کار با نرم‌افزار مدیریت پروژه \چبر{jira}
\پایان{itemize}

\زیرقسمت{زمان‌بندی جلسات}
جلسات مورد نیاز پروژه‌ی \PRJN{} در دو دسته‌ی کلّی جای می‌گیرند:
\شروع{itemize}
\فقره جلسات درون گروهی: این جلسات هر هفته و مطابق با آیین‌نامه‌ی گروه برگزار می‌شود و در آن اعضای گروه علاوه‌بر ارائه‌ی گزارش پیشرفت کارها، به تبادل نظر و اصلاح جهت پیشرفت پروژه می‌پردازند.
\فقره جلسات برون‌گروهی: این جلسات، بسته به درخواست ناظر پروژه و به صورت ماهیانه برگزار می‌شود. در این جلسات، مدیریت گروه گزارشی از پیشرفت پروژه ارائه می‌کند، و با توجه به بازخورد ناظر، مسیر انجام پروژه را اصلاح می‌کند. نتیجه‌ی این جلسات در جلسات درون‌گروهی منعکس خواهد شد.
\پایان{itemize}

\زیرقسمت{فواصل و شیوه‌ی ارائه‌ی گزارش‌ها}
علاوه بر گزارش‌هایی که در جلسات ارائه می‌شود و در بخش قبل به آن‌ها اشاره شد، اعضای گروه موظفند تا در فاز پیاده‌سازی تمامی فعالیت‌های خود را از طریق نرم‌افزار \چبر{issue tracker} مستندسازی کنند. این مستندات به صورت برخط در دسترس ناظر پروژه خواهد بود و در صورت نیاز، ایشان می‌تواند توسّط \چبر{issue tracker} با توسعه‌دهندگان ارتباط برقرار کنند.

همچنین، در فاز تحلیل و طراحی، دو مرحله ارائه‌ی \چبر{Use Case} و \چبر{ERD} و \چبر{DFD} نیز موجود است، که در این مراحل نیز ناظر می‌تواند نظر خود را به اطلاعات طراحان برساند تا بررسی و اعمال شوند.

\زیرقسمت{مدیریت ناسازگاری‌ها}
در صورت بروز ناسازگاری میان اعضای گروه و یا میان ناظر و گروه، مراحل زیر دنبال خواهند شد:
\شروع{enumerate}
\فقره تلاش اوّلیه برای متقاعد کردن تمامی اعضای گروه و ناظران، و جلب رضایت آن‌ها می‌باشد.
\فقره در صورت بی‌ثمر ماندن گام پیشین، رأی‌گیری میان حضّار انجام خواهد شد.
\پایان{enumerate}

\زیرقسمت{مدیریت گستره}
در صورت درخواست \چبر{feature}های جدید (چه از سوی اعضای گروه، و چه از سوی کارفرما)، جهت جلوگیری از رخداد \چبر{feature creep}، اعضای گروه جلسه‌ای خواهند داشت که در آن به بررسی مزایا و معایب اضافه شدن یک قابلیت به پروژه می‌پردازند. در حالت کلّی، زمانی با افزایش گستره موافقت خواهد شد که پیاده‌سازی آن ارزش افزوده‌ای را به‌همراه داشته باشد و شرایط پروژه (مانند بودجه و زمان‌بندی) را نقض نکند.

